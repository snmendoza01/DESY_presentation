%----------------------------------------------------------------------
% Missuse parts as chapters according to DESY PR
\part[Part slide]{Physics}
\makepart%

%----------------------------------------------------------------------
% Add a sample slide with some math, some columns etc.
\begin{frame}[t,label=intro]
    \frametitle{Problem Setup}
    \begin{itemize}[<+->]
        \setlength\itemsep{.5em}
        \item Time Independent Schrödiger Equation (\textbf{TISE}) 
        \[ E\ket{\psi} = \Hhat\ket{\psi}\]
        \item We approximate it using the variational principle~\ref{}
        \item Fundamentally: We need loss function to optimize
        \item Energy of approximation state $E_\Theta$ is always $\geq$ than groundstate $E_0$
        \begin{proof}
            Assuming normalization of $\ket{\psi_{\Theta}}$ and letting ${\{\ket{k}\}}_{k=1}^\infty$ be eigenstates of $\Hhat$
            \begin{equation}
                E_\Theta = \bra{\psi_{\Theta}}\Hhat\ket{\psi_{\Theta}} = 
                \sum_{m, n = 0}^\infty \bar{\alpha}_m \alpha_n 
                \underbrace{\bra{m}\Hhat\ket{n}}_{E_n\cdot \delta_{m,n}}
                = \sum_{n=0}^\infty |\alpha_n|^2 E_n \geq E_0 \sum_{n=0}^\infty |\alpha_n|^2  = E_0
                \label{eq:larger_than_groundstate_E}
            \end{equation}
        \end{proof}
    \end{itemize}
\end{frame}

\begin{frame}
    \frametitle{Variational Principle}
    \begin{itemize}
        \setlength\itemsep{0.9em}
        \item This principle applies to higher order eigenenergies
        \item Idea: Consider orthogonal subspace to $\ket{0}$ and reapply~\eqref{eq:larger_than_groundstate_E}
        \item Can be done for all space (if it is separable)
        \item More rigorous approach in~\ref{}
        \item $\Rightarrow$ must diagonalize ${[\widetilde{H}]}_{ij}=\bra{\varphi_i}\Hhat\ket{\varphi_j}$
                given arbitrary orthonormal basis ${\{\ket{\varphi_k}\}}_{k=1}^\infty$
        \item Can be an infinite dimensional matrix
        \item $\infty$ is a problem numerically
    \end{itemize}
\end{frame}

\begin{frame}
    \frametitle{Managing $\infty$}
    \begin{itemize}[<+->]
        \setlength\itemsep{0.8em}
        \item Define a truncation parameter $\nmax$
        \item Approximate state as \[\ket{\psi} \approx \sum_{k=0}^{\nmax} c_k \ket{\varphi_k}\]
        \item Becomes finite-dimensional problem
        \item Recall: Curse of dimensionality!
        \item $\nmax$ needed to converge to real results scales \textit{exponentially} with $d$
        \item Approach: define a more flexible `augmented basis' ${\{\ket{\varphi_i^A}\}}_i$ 
        \item Reduce $\nmax$ needed when augmented basis is optimized
    \end{itemize}
\end{frame}


% left aligned columns. Note that columns with smaller width are
% centred automatically
% \begin{columns}
%     \begin{column}[t]{0.48\textwidth}
%     Duis autem vel eum iriure dolor in hendrerit in vulputate velit
%     esse molestie consequat, vel illum dolore eu feugiat nulla
%     facilisis at vero eros et accumsan\ldots
%     \end{column}
%     \begin{column}[t]{0.48\textwidth}
%     Nam liber tempor cum soluta nobis eleifend option congue nihil
%     imperdiet \ldots
%     \end{column}
% \end{columns}

